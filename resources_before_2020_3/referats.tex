\newpage
\thispagestyle{empty}

\markboth{{\footnotesize{{\it \textbf{Рефераты}}}}} {{\footnotesize{{\it \textbf{Abstracts}}}}}

\addcontentsline{toc}{art}{Рефераты}
\addcontentsline{tec}{art}{Abstracts}

\parindent=0pt

\begin{tabular}{@{}l@{\hspace{5.6cm}}r}
{\sf\textbf{РЕФЕРАТЫ}} & {\sf\textbf{ИНФОРМАТИКА\; И\; МАТЕМАТИКА}}\\
\end{tabular}
\hrule height 1pt \vskip 2pt \hrule \vskip 0.5cm

%%%%%%%%%%%%%%% ABSTRACTS %%%%%%%%%%%%%%%

%%   Для русскоязычных и украиноязычных статей
%%       #1 - Первый автор (можно нескольких, а можно и всех) в формате Фамилия И. О.
%%       #2 - Все авторы в формате И. О. Фамилия
%%       #3 - Название статьи
%%       #4 - Первая страница
%%       #5 - Последняя страница
%%       #6 - УДК
%%       #7 - Аннотация на русском языке
%%       #8 - Ключевые слова на русском языке
\def\tvimRef#1#2#3#4#5#6#7#8{{\bf#1\;\;#3~/ #2~// \tvimname.~--- \tvimyear.~---
\tvimnumber.~--- C.\;\pageref{#4}\,--\,\pageref{#5}.}

\vspace{9pt}

{\textbf{УДК:~#6}}

\vspace{7pt}

#7

\vspace{3pt} {\small \textit{\textbf{Ключевые слова:} #8.}}}

%%   Для английских статей
%%       #1 - Author(s)
%%       #2 - Title
%%       #3 - First Page Ref
%%       #4 - Second Page Ref
%%       #5 - MSC2010
%%       #6 - Abstract on Russian
%%       #7 - Ключевые слова на русском языке
\def\tvimRefEn#1#2#3#4#5#6#7{{\bf#1\;\;\tvimyearen.\;\;#2. \textit{\tvimnameen},
\tvimnumberen, pp.\;\pageref{#3}\,--\,\pageref{#4}.}

\vspace{7pt}

{\textbf{MSC2010:~#5}}

\vspace{5pt}

#6

\vspace{3pt} {\small \textit{\textbf{Keywords:} #7.}}}

%#1 - margin before line
%#2 - margin after line
\def\abstractLine#1#2{\vskip #1 \hrule \vskip 2pt \hrule \vskip #2}
