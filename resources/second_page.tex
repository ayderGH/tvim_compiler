\def\baselinestretch{1}
\thispagestyle{empty}

\myinter=2pt

\parindent=0mm
{\scriptsize\sf
    \renewcommand{\arraystretch}{0}
    \begin{tabular}{l}
%        \textbf{\copyright\;КРЫМСКИЙ ФЕДЕРАЛЬНЫЙ УНИВЕРСИТЕТ им.~В.\;И.\;ВЕРНАДСКОГО}\\ 
        \textbf{УЧРЕДИТЕЛЬ~--- ФГАОУ ВО <<КРЫМСКИЙ ФЕДЕРАЛЬНЫЙ УНИВЕРСИТЕТ им.~В.\;И.\;ВЕРНАДСКОГО>>}\\
        
        \rule{0pt}{0.6cm}\\%
        \textbf{ЧЛЕНЫ РЕДКОЛЛЕГИИ}\\
    \end{tabular}
}

\vspace{0.3cm} {\renewcommand{\arraystretch}{0.4}
\begin{tabular}{ll}
{\qquad\scriptsize\sf \textbf{М.\;А.\;МУРАТОВ}} & {\scriptsize\sf \textbf{главный редактор}, профессор, доктор физико-математических наук}\\
%                                               & {\scriptsize\sf доктор физико-математических наук}\\

\rule{0pt}{4pt} & \\
{\qquad\scriptsize\sf \textbf{С.\;В.\;АБЛАМЕЙКО}}   & {\scriptsize\sf академик НАН Беларуси, профессор, доктор технических наук}\\

\rule{0pt}{4pt} & \\
{\qquad\scriptsize\sf \textbf{К.\;В.\;ВОРОНЦОВ}}   & {\scriptsize\sf профессор, доктор физико-математических наук}\\

\rule{0pt}{4pt} & \\
{\qquad\scriptsize\sf \textbf{Ю.\;И.\;ЖУРАВЛЕВ}}       & {\scriptsize\sf академик РАН, доктор физико-математических наук}\\

\rule{0pt}{4pt} & \\
{\qquad\scriptsize\sf \textbf{К.\;В.\;РУДАКОВ}}        & {\scriptsize\sf академик РАН, доктор физико-математических наук}\\

\rule{0pt}{4pt} & \\
{\qquad\scriptsize\sf \textbf{А.\;М.\;ГУПАЛ}}          & {\scriptsize\sf член-корреспондент НАНУ, доктор физико-математических наук}\\

\rule{0pt}{4pt} & \\
{\qquad\scriptsize\sf \textbf{А.\;Н.\;КАРАПЕТЯНЦ}}     & {\scriptsize\sf доцент, доктор физико-математических наук}\\

\rule{0pt}{4pt} & \\
{\qquad\scriptsize\sf \textbf{В.\;В.\;КРАСНОПРОШИН}}   & {\scriptsize\sf профессор, доктор технических наук}\\

\rule{0pt}{4pt} & \\
{\qquad\scriptsize\sf \textbf{Л.\;М.\;МЕСТЕЦКИЙ}}   & {\scriptsize\sf профессор, доктор технических наук}\\

\rule{0pt}{3pt} & \\
{\qquad\scriptsize\sf \textbf{А.\;Г.\;НАКОНЕЧНЫЙ}}     & {\scriptsize\sf профессор, доктор физико-математических наук}\\

\rule{0pt}{4pt} & \\
{\qquad\scriptsize\sf \textbf{А.\;А.\;САПОЖЕНКО}}      & {\scriptsize\sf профессор, доктор физико-математических наук}\\

\rule{0pt}{4pt} & \\
{\qquad\scriptsize\sf \textbf{В.\;В.\;СТАРОСТЕНКО}}      & {\scriptsize\sf профессор, доктор физико-математических наук}\\

\rule{0pt}{4pt} & \\
{\qquad\scriptsize\sf \textbf{В.\;Н.\;ЧЕХОВ}}          & {\scriptsize\sf профессор, доктор физико-математических наук}\\

\rule{0pt}{4pt} & \\
{\qquad\scriptsize\sf \textbf{В.\;И.\;ЧИЛИН}}          & {\scriptsize\sf профессор, доктор физико-математических наук}\\

\end{tabular}

\vspace{0.4cm} {\renewcommand{\arraystretch}{0.2}
  \begin{tabular}{l}
    {\scriptsize\sf \textbf{СЕКРЕТАРИАТ РЕДКОЛЛЕГИИ:}}\\\\
    \qquad{\scriptsize\sf к.\,ф.-м.\,н., доцент}~{\scriptsize\sf\textbf{А.\;С.\;АНАФИЕВ}}~--
        {\scriptsize\sf \textbf{ответственный редактор} (раздел <<Информатика>>)}\\\\
          \qquad{\scriptsize\sf к.\,ф.-м.\,н., доцент}~{\scriptsize\sf\textbf{В.\;И.\;ВОЙТИЦКИЙ}}~--
        {\scriptsize\sf \textbf{ответственный редактор} (раздел <<Математика и механика>>)}\\\\
		  \qquad{\scriptsize\sf к.\,ф.-м.\,н., доцент}~{\scriptsize\sf \textbf{В.\;Ф.\;БЛЫЩИК\phantom{Й}}}~--
        {\scriptsize\sf \textbf{редактор сайта журнала}}\\\\
        \qquad{\scriptsize\sf к.\,ф.-м.\,н., доцент}~{\scriptsize\sf \textbf{М.\;Г.\;КОЗЛОВА\phantom{Й}}}~--
        {\scriptsize\sf \textbf{ученый секретарь журнала}}
  \end{tabular}
}

\vspace{0.4cm}%
{\renewcommand{\arraystretch}{0}
  \begin{tabular}{l}
    {\scriptsize\sf \textbf{АДРЕС РЕДАКЦИИ:}}\\ \rule{0pt}{\myinter}\\%
    {\qquad\scriptsize\sf Крымский федеральный университет им.\;В.\;И.\;Вернадского}\\ \rule{0pt}{\myinter}\\
    {\qquad\scriptsize\sf пр-т Академика Вернадского, 4, г.\;Симферополь, Республика Крым, Российская Федерация, 295007} \\
  \end{tabular}
}

\vspace{0.3cm}%
{\renewcommand{\arraystretch}{0}
  \begin{tabular}{l}
    {\scriptsize\sf \textbf{ДЛЯ ПЕРЕПИСКИ:}}\\ \rule{0pt}{\myinter}\\%
    {\qquad\scriptsize\sf Таврический вестник информатики и математики}\\ \rule{0pt}{\myinter}\\
    {\qquad\scriptsize\sf Факультет математики и информатики, Таврическая академия КФУ им.\;В.\;И.\;Вернадского}\\ \rule{0pt}{\myinter}\\%
    {\qquad\scriptsize\sf пр-т Академика Вернадского, 4, г.\;Симферополь, Республика Крым, Российская Федерация, 295007}\\
  \end{tabular}
}

\vspace{0.3cm}%
{\renewcommand{\arraystretch}{0}
  \begin{tabular}{ll}
    {\qquad\scriptsize\sf Тел. гл.\;редактора: }        & {\scriptsize\sf +7 (3652) 63-75-42} \\ \rule{0pt}{\myinter} & \\%
    {\qquad\scriptsize\sf Тел. редакции: }            & {\scriptsize\sf +7 (3652) 602-466} \\ \rule{0pt}{\myinter} & \\%
    {\qquad\scriptsize\sf e-mail (гл.\;редактор): }     & {\scriptsize\sf vidonskoy@mail.ru} \\ \rule{0pt}{\myinter} & \\%
    {\qquad\scriptsize\sf e-mail (для переписки): }   & {\scriptsize\sf \tvimemail} \\ \rule{0pt}{\myinter} & \\%
    {\qquad\scriptsize\sf сайт журнала: }             & {\scriptsize\sf \tvimwww} \\ \rule{0pt}{\myinter} & \\%
  \end{tabular}
}

\vspace{0.3cm}%
{\scriptsize\sf
    \renewcommand{\arraystretch}{0}
        \begin{tabular}{l}
            \textbf{Журнал публикует оригинальные и обзорные статьи} \\ \rule{0pt}{\myinter} \\%
            \textbf{по вопросам теоретической и прикладной информатики и математики}\\ \rule{0pt}{\myinter} \\%
        \end{tabular}
}

\vspace{0.3cm} {\renewcommand{\arraystretch}{0}
  \begin{tabular}{l}
    {{\scriptsize\sf \textbf{Ведущие тематические разделы:}}}
  \end{tabular}
}

\enlargethispage{4\baselineskip}

\myinter=5pt
\vspace{0.1cm}%
{
\renewcommand{\arraystretch}{0}
{\scriptsize\sf
\begin{tabular}{l@{\hspace{1.65cm}}l}
	Функциональный анализ и его приложения      & Машинное обучение, распознавание и\\                                         & извлечение закономерностей\\
	
	\rule{0pt}{\myinter} & \\%

	Интегральные, дифференциальные уравнения    & Дедуктивные системы и базы знаний \\
	и динамические системы \phantom{W}          & \\
	
	\rule{0pt}{\myinter} & \\%

	Спектральные и эволюционные задачи          & Знаниеориентированные и гибридные\\
	                                            & математические модели принятия решений\\
	\rule{0pt}{\myinter} & \\

	Математические проблемы гидродинамики       & Синтез моделей принятия решений \\
                                                & при неполной начальной информации \\
                                                & \\%

	Дискретная оптимизация                      &  \\
	
	\rule{0pt}{\myinter} & \\
	
	Математическая логика, теория алгоритмов    & Вычислительная математика \\
	и теория сложности вычислений               & \\
\end{tabular}
}}


\vspace{0.5cm}%
\hrule
\vspace{1mm}
{
    \parindent=-1.5mm
    \renewcommand{\arraystretch}{1}
    \tiny\sf
    
    \begin{tabular}{p{\textwidth}}
        Печатается по решению научно-технического совета Крымского федерального университета имени~В.\,И.\,Вернадского. \\
        Протокол~\protocol
    \end{tabular}

}

%возвращаем исходные значения параметров
\parindent=\myparindent
\def\baselinestretch{1.2}
\renewcommand{\arraystretch}{1}